\documentclass[11pt]{article}
\usepackage{fullpage}
\renewcommand*\contentsname{Sum\'ario}
 
\begin{document}

\title{Trabalho 01 - Comunica\c{c}\~ao Entre Processos}
\author{Adailson Pinho dos Santos - 13/0140724\\
Vitor Nere Ara\'ujo Ribeiro - 13/0137413}
\date{}
\maketitle

\newpage

\tableofcontents

\newpage

\section{Introdu\c{c}\~ao}
	\paragraph{}	O presente documento visa descrever as ferramentas utilizadas no desenvolvimento do trabalho, quais foram os sistemas operacionais utilizados, qual ambiente de desenvolvimento, quais foram as entradas e sa\'idas dos softwares desenvolvidos e quais s\~ao as limita\c{c}\~oes dos software, este trabalho foi contru\'ido por uma dupla de estudantes da disciplina de Fundamentos de Arquitetura de Computadores da Universidade de Bras\'ilia. 
\section{Ambiente de desenvolvimento}
    \paragraph{}	Para a execu\c{c}\~ao do desenvolvimento das aplica\c{c}\~oes foram utilizados os Sistemas Operacionais Linux Mint 18.2 Sonya e Debian 9, sendo que ambos s\~ao distribui\c{c}\~oes populares do Linux. 
    \paragraph{}	Foi utilizada a ferramenta Make que funciona como um sistema de alvos e depend\^encias. Dessa forma, estar\'a definido dentro do arquivo Makefile quais arquivos ele ir\'a processar para realizar determinada tarefa. Sendo assim, utilizou-se a vers\~ao GNU Make 4.1 e GCC vers\~ao 5.4 nos sistemas GNU/Linux.
    \paragraph{}	Para o desenvolvimento da especifica\c{c}\~ao do presente trabalho, utilizou-se o Texmaker 4.4.1, sendo este um programa que serviu para compila\c{c}\~ao e exporta\c{c}\~ao do trabalho em PDF por meio da linguagem LaTeX.
\section{Constru\c{c}\~ao Quest\~ao 01}
	\subsection{Racioc\'inio L\'ogico}
	\subsection{Makefile}
	\subsection{Fun\c{c}\~ao Main}
	\subsection{Arquivos .h}
	\subsection{Arquivos .c}
\section{Constru\c{c}\~ao Quest\~ao 02}
	\subsection{Racioc\'inio L\'ogico}
	\subsection{Makefile}
	\subsection{Fun\c{c}\~ao Main}
\section{Resultados}
	\subsection{Quest\~ao 01}
	\subsection{Quest\~ao 02}

\end{document}